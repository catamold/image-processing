\documentclass[conference]{IEEEtran}
\IEEEoverridecommandlockouts
% The preceding line is only needed to identify funding in the first footnote. If that is unneeded, please comment it out.
\usepackage{cite}
\usepackage{amsmath,amssymb,amsfonts}
\usepackage{algorithmic}
\usepackage{graphicx}
\usepackage{textcomp}
\usepackage{xcolor}
\usepackage{enumitem}
\usepackage{physics}
\def\BibTeX{{\rm B\kern-.05em{\sc i\kern-.025em b}\kern-.08em
    T\kern-.1667em\lower.7ex\hbox{E}\kern-.125emX}}
\begin{document}

\title{Manual de utilizare\\
{\footnotesize \textsuperscript{*}Procesare de Imagini - 9 Mai 2019}
}

\author{\IEEEauthorblockN{Catalin Moldovan}
\IEEEauthorblockA{\textit{Grupa 30235} \\
Universitatea Tehnica din Cluj-Napoca \\
catalin.moldovan@student.utcluj.ro}
}

\maketitle

\begin{abstract}
Acest document descrie modul de utilizare al aplicatiei \textit{Recunoastere pe baza de semnaturi}
\end{abstract}

\section{Adaugare dependinte}
\paragraph{Se vor adauga urmatoarele cai in sectiunea \textbf{General} din proprietatile aplicatiei}
\begin{itemize}
	\item \$(SolutionDir)/Libraries/OpenCV-4.1.0/opencv/build/x64/vc15/bin
	\item \$(SolutionDir)/Libraries/install/include
	\item \$(SolutionDir)/Libraries/OpenCV-4.1.0/opencv/build/include
\end{itemize}

\paragraph{Se vor adauga urmatoarele cai in sectiunea \textbf{Linker} din proprietatile aplicatiei}
\begin{itemize}
	\item \$(SolutionDir)/Libraries/OpenCV-4.1.0/opencv/build/x64/vc15/lib
	\item \$(SolutionDir)/Libraries/install/x64/vc15/lib
\end{itemize}

\paragraph{Se vor adauga urmatoarele fisiere in sectiunea \textbf{Input} din proprietatile aplicatiei}
\begin{itemize}
	\item opencv\_world410d.lib
	\item opencv\_xfeatures2d410d.lib
\end{itemize}

\section{Crearea fisierelor}
In folderul \textbf{Semnaturi} se vor crea foldere (test1,test2,...) cu imagini de test, utilizate pentru antrenarea algoritmului, dupa urmatoarea regula: \textbf{image\_XXXX.jpg}, unde \textit{XXXX} reprezinta numarul imaginii de test (ex. 0001).

In aceeasi locatie se creeaza folderul \textit{input} ce contine imaginea de intrare pentru incadrarea acesteia in una din categoriile create la pasul anterior.

\section{Procesare imagini de test}
In fisierul \textbf{Program.cpp}, in variabila \textit{DATASET\_PATH} se introduce calea catre fisierul ce contine folderele create la pasul anterior, utilizand doua caractere despartitoare. Variabila \textit{TESTING\_PERCENT\_PER} contine numarul imaginii introduse in folderul \textit{input}.

Metoda \textbf{algoritmSemnatura} va utiliza algoritmul de clasificare K-means. Acesta presupune antrenarea prin imagini de test ce fac parte din anumite categorii, in cazul nostru semnaturi ale diferitelor persoane. Astfel se vor adauga metode \textbf{readGetImageDescriptors}("test[no]", [semnaturi], [cat]), unde \textit{no} reprezinta numarul folderului de test, \textit{semnaturi} numarul de imagini de test prezente in folderul respectiv si \textit{cat} reprezinta numarul categoriei incepand cu valoare 1 si continuand crescator in functie de numarul de metode adaugate.

Metoda \textbf{testData}("input", TESTING\_PERCENT\_PER, [cat]) reprezinta clasificarea imaginii de intrare cu categoriile de test incluse in variabila \textit{cat}.

Metoda \textbf{algoritmHist} va utiliza un algoritm de comparare a imaginilor de test cu cea de intrare prin histograma cu numar de acumulare redus. Se utilizeaza astfel media dintre numarul de pixeli negri de pe orizontala si verticala a fiecarei imagini de test cu cea de intrare. Se vor adauga numele folderelor de test: (test1,test2,...) in variabila \textbf{testImg[no]}, unde \textit{no} reprezinta numarul total de fisiere adaugate.

In pasul urmator se va introduce calea folderelor de test in metoda \textbf{processedImage} din variabila \textbf{imgTest} si calea imaginii de intrare din variabila \textbf{srcInput}.

\section{Rulare program}
Rularea programului se va efectua in modul \textbf{Debug} prin \textit{Local Windows Debugger}.

In procesul algoritmului K-means se vor citi, decta si transforma imaginile de test, urmand ca apoi sa se genereze histogramele si antrenarea SVM. Rezultatul final este redat prin obtinerea valorii de 0\% sau 100\% a acuratetii clasificarii pentru fiecare categorie de test. Pentru fiecare proces este calculat timpul de executie in secunde.

In procesul algoritmului de identificarii semnaturii prin histograma se reda in consola procentajul pentru fiecare categorie. In final procentajul cel mai ridicat, cu conditia ca acesta sa depaseasca 60\% reprezinta potrivirea semnaturii primite cu categoria din care face parte.

% Referinte
\begin{thebibliography}{00}
\bibitem{b1} ``DescriptorMatcher::match'', website: https://docs.opencv.org/2.4/modul es/features2d/doc/common\_interfaces\_of\_descriptor\_matchers.html
\bibitem{b2}``SVM::predict'', website: https://docs.opencv.org/3.1.0/d1/d73/tutorial\_in
troduction\_to\_svm.html
\bibitem{b3}``cv::xfeatures2d:;SIFT'', website: https://docs.opencv.org/3.4/d5/d3c/class
cv\_1\_1xfeatures2d\_1\_1SIFT.html
\bibitem{b4}``kmeans'', website: https://docs.opencv.org/3.4/d1/d5c/tutorial\_py\_kmeans
\_opencv.html
\end{thebibliography}
\vspace{12pt}

\end{document}
